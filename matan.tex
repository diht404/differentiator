\documentclass[12pt,a4paper,fleqn]{article}
\usepackage[utf8]{inputenc}
\usepackage[russian]{babel}
\usepackage[shortcuts,cyremdash]{extdash}
\usepackage{wrapfig}
\usepackage{floatflt}
\usepackage{lipsum}
\usepackage{concmath}
\usepackage{euler}
\usepackage{libertine}
\usepackage{amsfonts}
\usepackage{amsmath}
\usepackage{hyperref}

\oddsidemargin=6mm

\parindent=0pt
\parskip=8pt
\pagestyle{empty}
\usepackage[normalem]{ulem} % uline
\usepackage{mdframed}
\usepackage{amsthm}
\theoremstyle{definition}
\newtheorem{Def}{Def.}
\newtheorem{Ex}{Ex.}

\newenvironment{Sol}[1][]{\textcolor{blue}{\bfseries Решение. }}{}

\flushbottom
\begin{document}Решим элементарную задачу на дифференцирование, которую автор данного учебника решал еще в 5 классе.


$\log_{ 2.71828 }{sin( x )}$

Используя Wolfram легко получить, что 

$( x )'_{x} =  1 $

Применяя знания, полученные на прошлой лекции, читатель без труда получит, что 

$(sin( x ))'_{x} = cos( x ) *  1 $

Используя Wolfram легко получить, что 

$(\log_{ 2.71828 }{sin( x )})'_{x} = \frac{cos( x ) *  1 }{\log_{ 2.71828 }{ 2.71828 } * sin( x )}
$
 1 $

Очевидно, что 

$ 1  * sin( x ) =  1  * sin( x )$

Тут могла быть Ваша реклама. 

$\frac{cos( x ) *  1 }{ 1  * sin( x )}
 = \frac{cos( x ) *  1 }{ 1  * sin( x )}
$
cos( x )$

Примем без доказательства, что 

$ 1  * sin( x ) = sin( x )$

Доказательство будет дано в следующем издании учебника. 

$\frac{cos( x )}{sin( x )}
 = \frac{cos( x )}{sin( x )}
$


\textbf{Answer:}

$\frac{cos( x )}{sin( x )}
$


\textbf{Tangent equation at 1.2:}

$y = 0.38878 * x + -0.536916$


\textbf{Taylor of function}


Оставим доказательство данного факта читателю в качестве несложного упражнения. 

$( x )'_{x} =  1 $

Мне было лень доказывать этот факт.

$(sin( x ))'_{x} = cos( x ) *  1 $

В результате простых рассуждений можно получить 

$(\log_{ 2.71828 }{sin( x )})'_{x} = \frac{cos( x ) *  1 }{\log_{ 2.71828 }{ 2.71828 } * sin( x )}
$
 1 $

Доказательство будет дано в следующем издании учебника. 

$ 1  * sin( x ) =  1  * sin( x )$

Доказательство данного факта можно найти в \href{https://www.youtube.com/watch?v=dQw4w9WgXcQ}{видеолекции} 

$\frac{cos( x ) *  1 }{ 1  * sin( x )}
 = \frac{cos( x ) *  1 }{ 1  * sin( x )}
$
cos( x )$

Очевидно, что 

$ 1  * sin( x ) = sin( x )$

Нетрудно догадаться, что 

$\frac{cos( x )}{sin( x )}
 = \frac{cos( x )}{sin( x )}
$


\textbf{Answer:}

$\frac{cos( x )}{sin( x )}
$

Доказательство будет дано в следующем издании учебника. 

$( x )'_{x} =  1 $

Как рассказывали в начальной школе, 

$(sin( x ))'_{x} = cos( x ) *  1 $

Используя теорему 1000 из тома 7 главы 666 и лемму 42 из тома 13 главы 66 нетрудно получить, что 

$( x )'_{x} =  1 $

Применяя знания, полученные на прошлой лекции, читатель без труда получит, что 

$(cos( x ))'_{x} = ( -1  - sin( x )) *  1 $

Оставим доказательство данного факта читателю в качестве несложного упражнения. 

$(\frac{cos( x )}{sin( x )}
)'_{x} = \frac{( -1  - sin( x )) *  1  * sin( x ) - cos( x ) * cos( x ) *  1 }{sin( x ) * sin( x )}
$
 -1  - sin( x )$

Как рассказывали в начальной школе, 

$( -1  - sin( x )) * sin( x ) = ( -1  - sin( x )) * sin( x )$

Тут могла быть Ваша реклама. 

$cos( x ) = cos( x )$

Тут могла быть Ваша реклама. 

$cos( x ) = cos( x )$

Доказательство данного факта можно найти в \href{https://www.youtube.com/watch?v=dQw4w9WgXcQ}{видеолекции} 

$cos( x ) *  1  = cos( x )$

Нетрудно догадаться, что 

$cos( x ) * cos( x ) = cos( x ) * cos( x )$

Как рассказывали в начальной школе, 

$( -1  - sin( x )) * sin( x ) - cos( x ) * cos( x ) = ( -1  - sin( x )) * sin( x ) - cos( x ) * cos( x )$

Примем без доказательства, что 

$sin( x ) * sin( x ) = sin( x ) * sin( x )$

Используя Wolfram легко получить, что 

$\frac{( -1  - sin( x )) * sin( x ) - cos( x ) * cos( x )}{sin( x ) * sin( x )}
 = \frac{( -1  - sin( x )) * sin( x ) - cos( x ) * cos( x )}{sin( x ) * sin( x )}
$


\textbf{Answer:}

$\frac{( -1  - sin( x )) * sin( x ) - cos( x ) * cos( x )}{sin( x ) * sin( x )}
$

Примем без доказательства, что 

$( x )'_{x} =  1 $

В любом учебнике написано, что 

$(sin( x ))'_{x} = cos( x ) *  1 $

Как рассказывали в начальной школе, 

$( x )'_{x} =  1 $

Доказательство данного факта можно найти в \href{https://www.youtube.com/watch?v=dQw4w9WgXcQ}{видеолекции} 

$(sin( x ))'_{x} = cos( x ) *  1 $

Оставим доказательство данного факта читателю в качестве несложного упражнения. 

$(sin( x ) * sin( x ))'_{x} = cos( x ) *  1  * sin( x ) + sin( x ) * cos( x ) *  1 $

В результате простых рассуждений можно получить 

$( x )'_{x} =  1 $

Доказательство будет дано в следующем издании учебника. 

$(cos( x ))'_{x} = ( -1  - sin( x )) *  1 $

Как рассказывали в начальной школе, 

$( x )'_{x} =  1 $

Применяя знания, полученные на прошлой лекции, читатель без труда получит, что 

$(cos( x ))'_{x} = ( -1  - sin( x )) *  1 $

Применяя знания, полученные на прошлой лекции, читатель без труда получит, что 

$(cos( x ) * cos( x ))'_{x} = ( -1  - sin( x )) *  1  * cos( x ) + cos( x ) * ( -1  - sin( x )) *  1 $

(((Какой-то комментарий))) 

$( x )'_{x} =  1 $

Нетрудно догадаться, что 

$(sin( x ))'_{x} = cos( x ) *  1 $

Легко видеть, что 

$( x )'_{x} =  1 $

Применяя знания, полученные на прошлой лекции, читатель без труда получит, что 

$(sin( x ))'_{x} = cos( x ) *  1 $

В любом учебнике написано, что 

$( -1 )'_{x} =  0 $

Примем без доказательства, что 

$( -1  - sin( x ))'_{x} =  0  - cos( x ) *  1 $

Оставим доказательство данного факта читателю в качестве несложного упражнения. 

$(( -1  - sin( x )) * sin( x ))'_{x} = ( 0  - cos( x ) *  1 ) * sin( x ) + ( -1  - sin( x )) * cos( x ) *  1 $

Доказательство данного факта можно найти в \href{https://www.youtube.com/watch?v=dQw4w9WgXcQ}{видеолекции} 

$(( -1  - sin( x )) * sin( x ) - cos( x ) * cos( x ))'_{x} = ( 0  - cos( x ) *  1 ) * sin( x ) + ( -1  - sin( x )) * cos( x ) *  1  - ( -1  - sin( x )) *  1  * cos( x ) + cos( x ) * ( -1  - sin( x )) *  1 $

Применяя знания, полученные на прошлой лекции, читатель без труда получит, что 

$(\frac{( -1  - sin( x )) * sin( x ) - cos( x ) * cos( x )}{sin( x ) * sin( x )}
)'_{x} = \frac{(( 0  - cos( x ) *  1 ) * sin( x ) + ( -1  - sin( x )) * cos( x ) *  1  - ( -1  - sin( x )) *  1  * cos( x ) + cos( x ) * ( -1  - sin( x )) *  1 ) * sin( x ) * sin( x ) - (( -1  - sin( x )) * sin( x ) - cos( x ) * cos( x )) * (cos( x ) *  1  * sin( x ) + sin( x ) * cos( x ) *  1 )}{sin( x ) * sin( x ) * sin( x ) * sin( x )}
$
cos( x )$

Нетрудно догадаться, что 

$ 0  - cos( x ) =  -1  - cos( x )$

(((Какой-то комментарий))) 

$( -1  - cos( x )) * sin( x ) = ( -1  - cos( x )) * sin( x )$

Доказательство данного факта можно найти в \href{https://www.youtube.com/watch?v=dQw4w9WgXcQ}{видеолекции} 

$sin( x ) = sin( x )$

Мне было лень доказывать этот факт.

$ -1  - sin( x ) =  -1  - sin( x )$

В любом учебнике написано, что 

$cos( x ) = cos( x )$

В результате простых рассуждений можно получить 

$cos( x ) *  1  = cos( x )$

Доказательство будет дано в следующем издании учебника. 

$( -1  - sin( x )) * cos( x ) = ( -1  - sin( x )) * cos( x )$

Доказательство будет дано в следующем издании учебника. 

$( -1  - cos( x )) * sin( x ) + ( -1  - sin( x )) * cos( x ) = ( -1  - cos( x )) * sin( x ) + ( -1  - sin( x )) * cos( x )$

Зачем Вы читаете эти комментарии, в них нет никакого смысла... 

$sin( x ) = sin( x )$

Зачем Вы читаете эти комментарии, в них нет никакого смысла... 

$ -1  - sin( x ) =  -1  - sin( x )$

Как рассказывали в начальной школе, 

$( -1  - sin( x )) *  1  =  -1  - sin( x )$

Легко видеть, что 

$cos( x ) = cos( x )$

Нетрудно догадаться, что 

$( -1  - sin( x )) * cos( x ) = ( -1  - sin( x )) * cos( x )$

В результате простых рассуждений можно получить 

$cos( x ) = cos( x )$

Как рассказывали в начальной школе, 

$sin( x ) = sin( x )$

Очевидно, что 

$ -1  - sin( x ) =  -1  - sin( x )$

Примем без доказательства, что 

$( -1  - sin( x )) *  1  =  -1  - sin( x )$

Отсюда очевидно следует, что 

$cos( x ) * ( -1  - sin( x )) = cos( x ) * ( -1  - sin( x ))$

Мне было лень доказывать этот факт.

$( -1  - sin( x )) * cos( x ) + cos( x ) * ( -1  - sin( x )) = ( -1  - sin( x )) * cos( x ) + cos( x ) * ( -1  - sin( x ))$

Тут могла быть Ваша реклама. 

$( -1  - cos( x )) * sin( x ) + ( -1  - sin( x )) * cos( x ) - ( -1  - sin( x )) * cos( x ) + cos( x ) * ( -1  - sin( x )) = ( -1  - cos( x )) * sin( x ) + ( -1  - sin( x )) * cos( x ) - ( -1  - sin( x )) * cos( x ) + cos( x ) * ( -1  - sin( x ))$

(((Какой-то комментарий))) 

$sin( x ) * sin( x ) = sin( x ) * sin( x )$

Очевидно, что 

$(( -1  - cos( x )) * sin( x ) + ( -1  - sin( x )) * cos( x ) - ( -1  - sin( x )) * cos( x ) + cos( x ) * ( -1  - sin( x ))) * sin( x ) * sin( x ) = (( -1  - cos( x )) * sin( x ) + ( -1  - sin( x )) * cos( x ) - ( -1  - sin( x )) * cos( x ) + cos( x ) * ( -1  - sin( x ))) * sin( x ) * sin( x )$

Доказательство будет дано в следующем издании учебника. 

$sin( x ) = sin( x )$

Применяя знания, полученные на прошлой лекции, читатель без труда получит, что 

$ -1  - sin( x ) =  -1  - sin( x )$

(((Какой-то комментарий))) 

$( -1  - sin( x )) * sin( x ) = ( -1  - sin( x )) * sin( x )$

Очевидно, что 

$cos( x ) = cos( x )$

Нетрудно догадаться, что 

$cos( x ) = cos( x )$

Отсюда очевидно следует, что 

$cos( x ) * cos( x ) = cos( x ) * cos( x )$

Применяя знания, полученные на прошлой лекции, читатель без труда получит, что 

$( -1  - sin( x )) * sin( x ) - cos( x ) * cos( x ) = ( -1  - sin( x )) * sin( x ) - cos( x ) * cos( x )$

Очевидно, что 

$cos( x ) = cos( x )$

Мне было лень доказывать этот факт.

$cos( x ) *  1  = cos( x )$

Мне было лень доказывать этот факт.

$cos( x ) * sin( x ) = cos( x ) * sin( x )$

Легко видеть, что 

$cos( x ) = cos( x )$

Применяя знания, полученные на прошлой лекции, читатель без труда получит, что 

$cos( x ) *  1  = cos( x )$

В результате простых рассуждений можно получить 

$sin( x ) * cos( x ) = sin( x ) * cos( x )$

Очевидно, что 

$cos( x ) * sin( x ) + sin( x ) * cos( x ) = cos( x ) * sin( x ) + sin( x ) * cos( x )$

Зачем Вы читаете эти комментарии, в них нет никакого смысла... 

$(( -1  - sin( x )) * sin( x ) - cos( x ) * cos( x )) * (cos( x ) * sin( x ) + sin( x ) * cos( x )) = (( -1  - sin( x )) * sin( x ) - cos( x ) * cos( x )) * (cos( x ) * sin( x ) + sin( x ) * cos( x ))$

Очевидно, что 

$(( -1  - cos( x )) * sin( x ) + ( -1  - sin( x )) * cos( x ) - ( -1  - sin( x )) * cos( x ) + cos( x ) * ( -1  - sin( x ))) * sin( x ) * sin( x ) - (( -1  - sin( x )) * sin( x ) - cos( x ) * cos( x )) * (cos( x ) * sin( x ) + sin( x ) * cos( x )) = (( -1  - cos( x )) * sin( x ) + ( -1  - sin( x )) * cos( x ) - ( -1  - sin( x )) * cos( x ) + cos( x ) * ( -1  - sin( x ))) * sin( x ) * sin( x ) - (( -1  - sin( x )) * sin( x ) - cos( x ) * cos( x )) * (cos( x ) * sin( x ) + sin( x ) * cos( x ))$

Зачем Вы читаете эти комментарии, в них нет никакого смысла... 

$sin( x ) * sin( x ) = sin( x ) * sin( x )$

Доказательство данного факта можно найти в \href{https://www.youtube.com/watch?v=dQw4w9WgXcQ}{видеолекции} 

$sin( x ) * sin( x ) = sin( x ) * sin( x )$

(((Какой-то комментарий))) 

$sin( x ) * sin( x ) * sin( x ) * sin( x ) = sin( x ) * sin( x ) * sin( x ) * sin( x )$

Очевидно, что 

$\frac{(( -1  - cos( x )) * sin( x ) + ( -1  - sin( x )) * cos( x ) - ( -1  - sin( x )) * cos( x ) + cos( x ) * ( -1  - sin( x ))) * sin( x ) * sin( x ) - (( -1  - sin( x )) * sin( x ) - cos( x ) * cos( x )) * (cos( x ) * sin( x ) + sin( x ) * cos( x ))}{sin( x ) * sin( x ) * sin( x ) * sin( x )}
 = \frac{(( -1  - cos( x )) * sin( x ) + ( -1  - sin( x )) * cos( x ) - ( -1  - sin( x )) * cos( x ) + cos( x ) * ( -1  - sin( x ))) * sin( x ) * sin( x ) - (( -1  - sin( x )) * sin( x ) - cos( x ) * cos( x )) * (cos( x ) * sin( x ) + sin( x ) * cos( x ))}{sin( x ) * sin( x ) * sin( x ) * sin( x )}
$


\textbf{Answer:}

$\frac{(( -1  - cos( x )) * sin( x ) + ( -1  - sin( x )) * cos( x ) - ( -1  - sin( x )) * cos( x ) + cos( x ) * ( -1  - sin( x ))) * sin( x ) * sin( x ) - (( -1  - sin( x )) * sin( x ) - cos( x ) * cos( x )) * (cos( x ) * sin( x ) + sin( x ) * cos( x ))}{sin( x ) * sin( x ) * sin( x ) * sin( x )}
$


\textbf{Answer:}

$\log_{ 2.71828 }{sin( x )} = -inf+ inf \cdot x  -inf \frac{x^{2}}{2!}  -nan \frac{x^{3}}{3!}+ o(x^4)$


Отсюда очевидно следует, что 

$( x )'_{x} =  1 $

В результате простых рассуждений можно получить 

$(sin( x ))'_{x} = cos( x ) *  1 $

Используя теорему 1000 из тома 7 главы 666 и лемму 42 из тома 13 главы 66 нетрудно получить, что 

$(\log_{ 2.71828 }{sin( x )})'_{x} = \frac{cos( x ) *  1 }{\log_{ 2.71828 }{ 2.71828 } * sin( x )}
$
 1 $

Доказательство данного факта можно найти в \href{https://www.youtube.com/watch?v=dQw4w9WgXcQ}{видеолекции} 

$ 1  * sin( x ) =  1  * sin( x )$

Нетрудно догадаться, что 

$\frac{cos( x ) *  1 }{ 1  * sin( x )}
 = \frac{cos( x ) *  1 }{ 1  * sin( x )}
$
cos( x )$

Очевидно, что 

$ 1  * sin( x ) = sin( x )$

Оставим доказательство данного факта читателю в качестве несложного упражнения. 

$\frac{cos( x )}{sin( x )}
 = \frac{cos( x )}{sin( x )}
$


\textbf{Answer:}

$\frac{cos( x )}{sin( x )}
$

Как рассказывали в начальной школе, 

$( x )'_{y} =  0 $

Зачем Вы читаете эти комментарии, в них нет никакого смысла... 

$(sin( x ))'_{y} = cos( x ) *  0 $

Как рассказывали в начальной школе, 

$(\log_{ 2.71828 }{sin( x )})'_{y} = \frac{cos( x ) *  0 }{\log_{ 2.71828 }{ 2.71828 } * sin( x )}
$
 1 $

Оставим доказательство данного факта читателю в качестве несложного упражнения. 

$ 1  * sin( x ) =  1  * sin( x )$

Отсюда очевидно следует, что 

$\frac{cos( x ) *  0 }{ 1  * sin( x )}
 = \frac{cos( x ) *  0 }{ 1  * sin( x )}
$
 0 $

В результате простых рассуждений можно получить 

$ 1  * sin( x ) = sin( x )$

Используя Wolfram легко получить, что 

$\frac{ 0 }{sin( x )}
 =  0 $


\textbf{Answer:}

$ 0 $

Доказательство данного факта можно найти в \href{https://www.youtube.com/watch?v=dQw4w9WgXcQ}{видеолекции} 

$( x )'_{z} =  0 $

Тут могла быть Ваша реклама. 

$(sin( x ))'_{z} = cos( x ) *  0 $

Примем без доказательства, что 

$(\log_{ 2.71828 }{sin( x )})'_{z} = \frac{cos( x ) *  0 }{\log_{ 2.71828 }{ 2.71828 } * sin( x )}
$
 1 $

Тут могла быть Ваша реклама. 

$ 1  * sin( x ) =  1  * sin( x )$

(((Какой-то комментарий))) 

$\frac{cos( x ) *  0 }{ 1  * sin( x )}
 = \frac{cos( x ) *  0 }{ 1  * sin( x )}
$
 0 $

Оставим доказательство данного факта читателю в качестве несложного упражнения. 

$ 1  * sin( x ) = sin( x )$

Доказательство данного факта можно найти в \href{https://www.youtube.com/watch?v=dQw4w9WgXcQ}{видеолекции} 

$\frac{ 0 }{sin( x )}
 =  0 $


\textbf{Answer:}

$ 0 $


\textbf{Answer:} 

$ f'_x = \frac{cos( x )}{sin( x )}
$

$f'_y =  0 $

$f'_z =  0 $

$f'(3, 1, 2) = (-7.01525, 0, 0)$

$|f'(3, 1, 2)| = 3.74166$


\end{document}
