\documentclass[12pt,a4paper,fleqn]{article}
\usepackage[utf8]{inputenc}
\usepackage[russian]{babel}
\usepackage[shortcuts,cyremdash]{extdash}
\usepackage{wrapfig}
\usepackage{floatflt}
\usepackage{lipsum}
\usepackage{concmath}
\usepackage{euler}
\usepackage{libertine}
\usepackage{amsfonts}
\usepackage{amsmath}
\usepackage{hyperref}

\oddsidemargin=6mm

\parindent=0pt
\parskip=8pt
\pagestyle{empty}
\usepackage[normalem]{ulem} % uline
\usepackage{mdframed}
\usepackage{amsthm}
\theoremstyle{definition}
\newtheorem{Def}{Def.}
\newtheorem{Ex}{Ex.}

\newenvironment{Sol}[1][]{\textcolor{blue}{\bfseries Решение. }}{}

\flushbottom
\begin{document}Решим элементарную задачу на дифференцирование, которую автор данного учебника решал еще в 5 классе.


${ 3 }^{sin(\log_{ 2.71828 }{ x  +  10 })}$

Оставим доказательство данного факта читателю в качестве несложного упражнения. 

$( 10 )' =  0 $

Используя Wolfram легко получить, что 

$( x )' =  0 $

Тут могла быть Ваша реклама. 

$( x  +  10 )' =  0  +  0 $

Используя теорему 1000 из тома 7 главы 666 и лемму 42 из тома 13 главы 66 нетрудно получить, что 

$(\log_{ 2.71828 }{ x  +  10 })' = \frac{ 0  +  0 }{\log_{ 2.71828 }{ 2.71828 } * ( x  +  10 )}
$

Очевидно, что 

$(sin(\log_{ 2.71828 }{ x  +  10 }))' = cos(\log_{ 2.71828 }{ x  +  10 }) * \frac{ 0  +  0 }{\log_{ 2.71828 }{ 2.71828 } * ( x  +  10 )}
$

Как рассказывали в начальной школе, 

$({ 3 }^{sin(\log_{ 2.71828 }{ x  +  10 })})' = \log_{ 2.71828 }{ 3 } * cos(\log_{ 2.71828 }{ x  +  10 }) * \frac{ 0  +  0 }{\log_{ 2.71828 }{ 2.71828 } * ( x  +  10 )}
 * { 3 }^{sin(\log_{ 2.71828 }{ x  +  10 })}$
 1 $

Оставим доказательство данного факта читателю в качестве несложного упражнения. 

$ x  +  10  =  x  +  10 $

Применяя знания, полученные на прошлой лекции, читатель без труда получит, что 

$\log_{ 2.71828 }{ x  +  10 } = \log_{ 2.71828 }{ x  +  10 }$

Очевидно, что 

$cos(\log_{ 2.71828 }{ x  +  10 }) = cos(\log_{ 2.71828 }{ x  +  10 })$

Доказательство данного факта можно найти в \href{https://www.youtube.com/watch?v=dQw4w9WgXcQ}{видеолекции} 

$ 0  +  0  =  0 $

(((Какой-то комментарий))) 

$\log_{ 2.71828 }{ 2.71828 } =  1 $

Тут могла быть Ваша реклама. 

$ x  +  10  =  x  +  10 $

В результате простых рассуждений можно получить 

$ 1  * ( x  +  10 ) =  1  * ( x  +  10 )$

Применяя знания, полученные на прошлой лекции, читатель без труда получит, что 

$\frac{ 0 }{ 1  * ( x  +  10 )}
 = \frac{ 0 }{ 1  * ( x  +  10 )}
$

Зачем Вы читаете эти комментарии, в них нет никакого смысла... 

$cos(\log_{ 2.71828 }{ x  +  10 }) * \frac{ 0 }{ 1  * ( x  +  10 )}
 = cos(\log_{ 2.71828 }{ x  +  10 }) * \frac{ 0 }{ 1  * ( x  +  10 )}
$

(((Какой-то комментарий))) 

$ 1  * cos(\log_{ 2.71828 }{ x  +  10 }) * \frac{ 0 }{ 1  * ( x  +  10 )}
 =  1  * cos(\log_{ 2.71828 }{ x  +  10 }) * \frac{ 0 }{ 1  * ( x  +  10 )}
$

(((Какой-то комментарий))) 

$ x  +  10  =  x  +  10 $

Отсюда очевидно следует, что 

$\log_{ 2.71828 }{ x  +  10 } = \log_{ 2.71828 }{ x  +  10 }$

(((Какой-то комментарий))) 

${ 3 }^{sin(\log_{ 2.71828 }{ x  +  10 })} = { 3 }^{sin(\log_{ 2.71828 }{ x  +  10 })}$

Нетрудно догадаться, что 

$ 1  * cos(\log_{ 2.71828 }{ x  +  10 }) * \frac{ 0 }{ 1  * ( x  +  10 )}
 * { 3 }^{sin(\log_{ 2.71828 }{ x  +  10 })} =  1  * cos(\log_{ 2.71828 }{ x  +  10 }) * \frac{ 0 }{ 1  * ( x  +  10 )}
 * { 3 }^{sin(\log_{ 2.71828 }{ x  +  10 })}$
 x  +  10 $

Оставим доказательство данного факта читателю в качестве несложного упражнения. 

$\frac{ 0 }{ x  +  10 }
 =  0 $

Используя Wolfram легко получить, что 

$cos(\log_{ 2.71828 }{ x  +  10 }) *  0  =  0 $

Примем без доказательства, что 

$ 1  *  0  =  0 $

Используя Wolfram легко получить, что 

$ x  +  10  =  x  +  10 $

Тут могла быть Ваша реклама. 

$\log_{ 2.71828 }{ x  +  10 } = \log_{ 2.71828 }{ x  +  10 }$

Доказательство данного факта можно найти в \href{https://www.youtube.com/watch?v=dQw4w9WgXcQ}{видеолекции} 

${ 3 }^{sin(\log_{ 2.71828 }{ x  +  10 })} = { 3 }^{sin(\log_{ 2.71828 }{ x  +  10 })}$

Зачем Вы читаете эти комментарии, в них нет никакого смысла... 

$ 0  * { 3 }^{sin(\log_{ 2.71828 }{ x  +  10 })} =  0 $

\end{document}
